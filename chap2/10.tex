\subsection{}
$X = \{1,2,\ldots,n\}$ とする.

\paragraph*{証明1}

\paragraph*{証明2(「離散数学への招待」による鎖の構成)}
$S\subset X$ に対して,文字列 $f(S)$ を対応させる.
$i$ 文字目に,$i\in S$ であるか否かに応じて (, ) をあてる.

\begin{tabular}{|l|c|}
\hline
$S$ & $f(S)$ \\\hline
$2,3,5,9$ & ) \textcolor{red}{( ( ) ( ) )} ) \textcolor{red}{( )} ) )\\\hline
$2,3,5,9,12$ &  ) \textcolor{red}{( ( ) ( ) )} ) \textcolor{red}{( )} ) (\\\hline
$2,3,5,9,11,12$ &  ) \textcolor{red}{( ( ) ( ) )} ) \textcolor{red}{( )} ( (\\\hline
$2,3,5,8,9,11,12$ &  ) \textcolor{red}{( ( ) ( ) )} ( \textcolor{red}{( )} ( (\\\hline
$1,2,3,5,8,9,11,12$ &  ( \textcolor{red}{( ( ) ( ) )} ( \textcolor{red}{( )} ( (\\\hline
\end{tabular}

$f(S)$ における括弧が対応つく部分・対応のつき方が同一であるような $S$ を集める.
括弧列の対応箇所,対応方法を固定した場合,$f(S)$ の残りの部分の選択肢は $)))\cdots ))(((\cdots ($ のように,2種の括弧の境界をどこに置くかしかない.
よって鎖になること,また $S$ として $k, k+1, \ldots, n-k$ 元集合がひとつずつ含まれることが分かる.

特に,任意の鎖が $\lfloor \frac{n}{2}\rfloor$ 元部分集合を唯一含むような鎖に分解できた.