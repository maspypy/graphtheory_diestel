\subsection{}
$G=(V, E) = (A\amalg B, E)$ を二部グラフとする.
$x\leq x$ の他に,$ab\in E$ となる $a\in A, b\in B$ に対して $a\leq b$ と定めることで,$G$ を poset と見なす.
\begin{enumerate}
 \item $G$ の最大マッチングの大きさを $n_1$.
 \item $G$ の最小頂点カバーの大きさを $n_2$.
 \item $G$ の最小鎖被覆の大きさを $n_3$.
 \item $G$ の最大反鎖の大きさを $n_4$ とする.
\end{enumerate}

$n_2 = |V| - n_4$ を示す.
$S$ が頂点カバーであることと,$ab\in E\implies a\in S \text{or} b\in S$ は同値.
対偶をとって $A = G\setminus S$ とすることで,この条件は
$a\in A \text{and} b\in B\implies ab\notin E$ と書き換えられる.
これは,$A$ が反鎖であることと同値.したがって $n_2 = |V| - n_4$ を得る.

$n_1 = |V| - n_3$ を示す.$G$ の鎖への分解とは,$1$ 点または辺への分解に他ならない.
最小鎖被覆は辺を最も多く選んだ場合,したがって最大マッチングの場合に実現されるので,$n_1 = |V| - n_3$ が分かる.

したがって,$n_3=n_4\implies n_1=n_2$ であるから,Dilworth の定理は K\"{o}nig の定理を特殊ケースとして含んでいる.