\subsection{}
$\N\times \N$ に順序 $\leq$ を,$(x_1,x_2)\leq (y_1,y_2)\iff x_1\leq y_1 \text{かつ} x_2\leq y_2$ により定める.

反鎖 $A$ をひとつとりにおいて第 1 成分が最小のものを $(x_1,x_2)$ とすると,$(y_1,y_2)\in A$ に対して $y_2 < x_2$ が成り立つ.
特に第 $2$ 成分としてありうる値は有限通り.
また同一の第 $2$ 成分を持つ点は高々ひとつしかないので,$|A| < \infty$ が分かる.

$k$ を定数として $k$ 元からなる反鎖 $\{(x,y)\mid x+y=k+1\}$ を考えることで,$k$ 個未満の鎖で被覆できないことも分かる.
