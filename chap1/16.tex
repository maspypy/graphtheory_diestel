\subsection{}
頂点数$n$、最小次数$2k$のグラフ$G$が$k+1$-辺連結の部分グラフを持たないと仮定(※)して背理法で示す.
二分木$T$と、$T$の頂点を$G$の部分グラフに送る写像$f$を考える.
$T$と$f$は次の3つの条件を満たすとする.
\begin{itemize}
 \item 頂点$v$が木$T$の根ならば$f(v)=G$.
 \item 頂点$v$が木$T$の葉ならば$f(v)$の位数は1.
 \item 頂点$v$の子が$p,q$ならば$V(f(p))$と$V(f(q))$は$V(f(v))$の分割であり、$V(f(p))$と$V(f(q))$間の辺は$f(v)$に$k$個以下しか存在しない.
\end{itemize}

仮定(※)よりこのような$T$と$f$が取れる. 
$G$を$T$の各葉$v$に対応する$f(v)$にするために取り去った辺は合計で高々$k(n-1)$個である。一方、仮定より辺は少なくとも$kn$個ある. 
従って矛盾. 
