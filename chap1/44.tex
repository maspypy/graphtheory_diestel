\subsection{}
\begin{enumerate}
 \item
 \begin{enumerate}
  \item カットに延長できるならば奇サイクルを含まないこと.カットが奇サイクルを含まないことを示せばよい.
  任意のカットとサイクルは直交することから,カットに含まれるサイクル長は偶数に限られるのでよい.
  \item 奇サイクルを含まないならばカットに延長できること.奇サイクルを含まないならば二部グラフなので,ある分割 $V=A\perp B$ に対して
  $F\subset E(A,B)$ が成り立つ.$E(A,B)$ が求めるカットである.
 \end{enumerate}
 \item
 \begin{enumerate}
  \item サイクル空間の元に延長できるならば奇カットを含まないこと.これは (i)(a) と同様である.
  \item $F$ が奇カットを含まない辺集合として,$F$ がサイクル空間の元に延長できることを示す.
  
  $A$ を $G-F$ の連結成分のひとつとする.$F_0 = E(A,V\setminus A)$ とすると,$F_0$ は $F$ に含まれるカットであるから,
  仮定により $|F_0|$ は偶数である.$\sum_{v\in A} \deg_G(v) = \sum_{v\in A} \deg_{A}(v) + |F_0|$  より,$\sum_{v\in A}\deg_G(v)$ は偶数.
  よって,$A$ は奇点を偶数個持つ(したがって任意の連結成分はそうである).
  
  $A$ 内の奇点を $a_1,\ldots, a_{2n}$ とする.$A$ は $G-F$ の連結成分であるから,$F$ の辺を使わないパス $P_i$ であって $a_{2i-1}$ と $a_{2i}$ を結ぶものが存在する.
  $F_A = P_1 \mathrm{xor} P_2 \mathrm{xor} \cdots \mathrm{xor} P_n$ とする.$F_A$ は $F$ の辺を使っておらず,$A$ の各頂点は奇点ならば奇数個,偶点ならば偶数個 $F_A$ の辺と接続する.
  
  $G-F$ の各連結成分 $A$ に対して上のような $F_A$ を作って,$F_0 = E(G)\setminus (\bigcup_{A}F_A)$ とおく.
  $F_A \cap F = \emptyset$ であったから,$F_0$ は $F$ を含む.さらに,任意の $v\in V(G)$ に対して $\deg_{F_0}(v)$ は偶数である.したがって $F_0\in \mathcal{C}(G)$ が成り立ち,これが求めるものである.
 \end{enumerate}
\end{enumerate}