\subsection{}
$G$が連結であるとき,適当に $r$ をとったとき,$\{E(v)\mid v\neq r\}$ が基底であることを確かめる.
カット空間は $|G|-1$ 次元(Thm1.9.5)なので,線形独立性を示せばよい.

$\sum_{v\neq r}a_vE(v) = 0$ ($a_v\in \F_2$)を仮定する.
根に隣接する頂点 $v$ に対して,辺 $rv$ カットが $E(v)$ のみであることから,$a_v = 0$ が分かる.
同様にして親から順番に係数が $0$ に決まっていって示せる.

$G$ が連結でないときは,連結成分ごとに頂点 $r_1, \ldots, r_k$ を選ぶと $\{E(v)\mid v\neq r_i\}$ が基底になる.