\subsection{}
Oreの定理の証明を少し改変すればよい.

$|G| = n$ とする.長さが最大のパス $P = v_0v_1\ldots v_m$ (長さ $m$)をとる.$m\geq 2\delta(G)$ であれば示すことはない.$m < 2\delta(G)$ を仮定する.

まず,長さ $m+1$ のサイクルの存在を示す.
$A = \{1\leq i\leq m\mid v_i \in N(v_0)\}$, 
$B = \{0\leq i < m\mid v_i \in N(v_m)\}$, 
$B+ = \{i + 1\mid i \in B\}$ とおく.
$A, B_+\subset \{1,\ldots, m\}$ である.

$P$ の最大性の仮定より,$v_0, v_m$ と隣接する頂点は $v_i$ のいずれかであるから,$|A| = \deg(v_0) \geq \delta(G)$ である.
同様に $|B_+| = |B| \geq \delta(G)$ である.$|A| + |B_+| \geq 2\delta(G) > m \geq |A\cup B_+|$ であるから,$A$ と $B_+$ は交わる.

$i \in A\cap B_+$ とする.$v_0Pv_{i-1}v_nPv_iv_1$ が長さ $m+1$ のサイクルとなる.以上で長さ $m+1$ のサイクルの存在が示された.

$m + 1 = |G|$ であれば,このサイクルが求めるものである.そうでない場合,$G$ の連結性よりサイクルと隣接する新たな頂点が存在し,
長さ $m+1$ のパスが出来てしまう.最大性の仮定に矛盾.
