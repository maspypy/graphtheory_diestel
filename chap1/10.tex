\subsection{}
中心を根とするBFS木を観察する.深さ $i$ の部分の頂点集合を $D_i$ と書く.
$i \leq k$ とすると,$v\in D_i$ が存在.$N(v) \subset D_{i-1}\cup D_i\cup D_{i+1}$ から,
$|D_{i-1}| + D_i + D_{i+1} > 1 + d$ が分かる.

$|G| = (D_0+D_1+D_2) + (D_3+D_4+D_5)+\cdots$ という要領で評価すれば,$|G| > (1+d)\lfloor\frac{n}{e}\rfloor$ が得られる.

逆の評価:$D_{3i+1}$ が $d$ 点集合で,$D_{3i}$, $D_{3i+2}$ が $1$ 点集合となるように作れば最小次数の条件を満たす.