\subsection{}
$f(k) = 4k$ が条件を満たすことを示す.

$G$ が $\frac{|E|}{|V|} \geq 4k$ を満たすとする.
各頂点を赤,青で塗り分け $V = R\amalg B$ に対して,$E' = \{e = xy\mid x\in R, y\in B\}$ とすると,$G' = (V, E')$ は二部グラフである.
この二部グラフであって,最も生き残る辺の数 $|E'|$ が多いものをとる.

あらゆる塗り分けに対する生き残る辺の数の平均値は,$\frac12 |E|$ である(辺ごとの生存確率を足す).よって$|E'| \geq \frac12 |E|$ である.
$G'$ の平均次数は $2k$ 以上.よって Prop~1.2.2 より,その部分グラフであって最小次数が $k$ 以上のもの $H$ が存在.$H$ が条件を満たす.

 