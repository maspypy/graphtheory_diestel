\subsection{}
\begin{itemize}
 \item (i) $\implies$ (ii):一意性のみが問題.相異なるパス $P$, $Q$ があったとしてサイクルの存在を示す.

 $|P|$ で帰納法をする.
 $P = v_0v_1\ldots v_n$ とするとき,$v_i \in Q$ となる最小の $i\geq 1$ をとる.$i = n$ なら容易にサイクルが得られる.
 $P_1 = v_0Pv_i$, $P_2 = v_iPv_n$,  $Q_1 = v_0Qv_i$, $Q_2 = v_iQv_n$ とすると,$P_1\neq Q_1$ または $P_2\neq Q_2$ であるから,
 どちらかに対して帰納法の仮定が使えてサイクルが得られる.
 \item (ii) $\implies$ (iii):連結性はよい.$T-e$ が連結でないことを示せばよい.
 $e=xy$ に対して $T-e$ が連結だとすると,$x$ やら $y$ へのパス $P$ が存在し,$T$ において $P, e$ が相異なる $xy$ 間のパスとなって矛盾.
 \item (iii) $\implies$ (iv):$T+xy$ がサイクルを含むことは,$x$ から $y$ へのパスと $xy$ を合わせたものがサイクルなので.
 $T$ がサイクルを含まないことは,サイクルを含むとするとサイクルの $1$ 辺を除いても連結なので.
 \item (iv) $\implies$ (i):連結性を示せばよい.$x, y$ が隣接していないとき,$T+xy$ のサイクル $C$ に注目.
 $T$ はサイクルを含まないので $C$ は $xy$ を含みので,$x$ から $y$ への $T$ 上のパスが得られる.
\end{itemize}
