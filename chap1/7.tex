\subsection{}
$g = 2k+1$ あるいは $g = 2k + 2$ とおく.
$g$ が奇数ならば $1$ 点,偶数ならば隣接する $2$ 点を固定.
これらの点から距離 $n$ の点集合を $D_n$ とする.

$1\leq n < k$ ならば $v\in D_n$ の近傍 $N(v)$ は $D_{n-1}\cup D_n \cup D_{n+1}$ に含まれる.
$D_{n-1}$ に少なくとも $1$ 点の近傍がある.$D_{n-1}\cup D_n$ に $2$ 点あると $g$ より小さな閉歩道ができて矛盾.
$N(v)$ の元は,唯一の $D_{n-1}$ の点および $D_{n+1}$ の点からなることが分かる.

特に,$d-1$ 個以上の $D_{n+1}$ の点と接続する.また再び内周の議論から $u,v \in D_n$ に対して
$N(u)\cap N(v) = \emptyset$ がいえるのでこれらの点は $v$ ごとにdisjoint.
$|D_{n+1}|\geq (d-1) |D_n|$がいえる.

あとは $D_0$ が $1$, $2$ 点であることを使って下から評価すればできる.

後半の主張(平均次数の場合)は,1.2.2を使って最小次数の大きな部分グラフに注目すればよい.
