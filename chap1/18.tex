\subsection{}
(Hints では $G = K^{k+1}$ がギリギリであるような書かれ方がされていましたが,この証明のどこかおかしいですか?)


$b = -k + 1$ が最小であることを示す.つまり,
\begin{itemize}
 \item $\|G\| = k|G| - k$ かつ $G$ が $k$ 辺連結部分グラフを持たないものが存在する.
 \item $\|G\| > k|G| - k$ ならば,$G$ は $k$ 辺連結部分グラフを持つ.
\end{itemize}
を示せばよい.

(i) は,$|G| = 1$ の場合が満たす.

(ii) を示す.$|G|$ に関する帰納法.$|G|=1$ の場合はよい.
$|G|\geq 2$ かつ $\|G\| > k|G| - k$ であるとする.$G$ そのものが $(k+1)$ 辺連結ならば示すことはない.
$G$ が $(k+1)$ 辺連結でないとする.このとき,空でない頂点集合 $A, B$ であって,$V(G) = A\amalg B$ かつ $|E(A,B)|\leq k$ となるものが存在する.

$\|G\| = \|G[A]\| + \|G[B]\| + |E(A,B)| \leq \|G[A]\| + \|G[B]\| + k$ かつ $|G| = |A| + |B|$ である.
\begin{align*}
 0 &< \|G\| - k|G| + k \leq (\|G[A]\| + \|G[B]\| + k) - k(|A|+|B|) + k\\
   & = (\|G[A]\| - k|A|+k) + (\|G[B]\| - k|B|+k) 
\end{align*}
であるから,$\|G[A]\| > k|A|-k$ または $\|G[B]\| > k|B|-k$ が成り立つ.よって帰納法の仮定より,$G[A]$ または $G[B]$ が
$(k+1)$ 辺連結部分グラフを持つ.それは $G$ の部分グラフでもあるから示された.
