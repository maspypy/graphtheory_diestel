\subsection{}
まず $F=E(A, B)$ かつ $A$ が非連結であるとして,極小でないことを示す.$F$ がより小さなカットを含むことを示せばよい.

$A$ の連結成分のひとつを $A_1$ とし,$A_2 = A\setminus A_1$ とする.
$A_1, A_2$ の間に辺が存在しないので $E(A_1, B) = E(A_1, A_2\cup B)$, $E(A_2, B) = E(A_2, A_1\cup B)$ が成り立ちこれらはカットである.
$G$ の連結性よりこれらは空ではない.$F$ は $F_1, F_2$ に分割されることは明らか.

以上により,$E(A, B)$ がボンドならば $A$ は連結であることが示された.

逆に,空でない連結集合 $A, B$ への分割に対して $F = E(A, B)$ がボンドであることを示す.
$F$ がより小さなカット $F'=E(C,D)\subset F$ を含むとする.$A$ の点同士を含む辺は $F'$ に含まれない.
よって、$A$ 全体が $C$, $D$ でどちらかに完全に含まれる.$B$ についても同様で,結局 $F' = E(A, B)$ が言える.