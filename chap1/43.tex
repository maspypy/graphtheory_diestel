\subsection{}
\begin{enumerate}
 \item ボンド $F$ が,任意の全域木と交わる極小辺集合であること.
 
 $F$ はカットである($G-F$ は非連結である)ことから,任意の全域木と交わることは分かる.
 極小性を示す.$e\in F$ として,$F-e$ と交わらない全域木の存在を示せばよい.$e\in F$ として,$F\cap T = \{e\}$ となる全域木 $T$ の存在を示せばよい.
 
 $F = E(A, B)$ とすると,$F$ はボンドなので $A, B$ は連結(問題1.36).$A$ の全域木 $T_A$, $B$ の全域木 $T_B$ をとって
 $T = T_A + T_B + e$ とすれば,$T$ は全域木であり,$T\cap F = T\cap E(A,B) = \{e\}$ が成り立つのでよい.
 
 \item 任意の全域木と交わる極小辺集合 $F$ が,ボンドであること.
 
 まず $G\setminus F$ は非連結であることを示す.そうでないとすると,$G\setminus F$ の全域木 $T$ がとれる.
 これは $G$ の全域木でもあり,$F$ と交わらないので仮定に矛盾する.
 
 $F$ からひとつずつ辺を捨てていくと,$G\setminus F$ の連結成分は広義単調に減少し,$F = \emptyset$ のとき連結成分ひとつであるから,
 $F_0\subset F$ であって,$G\setminus F_0$ の連結成分がちょうど $2$ つであるものが存在する.
 $G\setminus F_0$ の連結成分を $A, B\subset V$ とする.$E(A,B)\subset F_0$ である.
 $G$ の任意の全域木は $E(A, B)$ の元と交わるので,$F_0$ と交わる.$F$ の極小性より $F=F_0$ である.
 よって,$A,B$ は連結かつ $E(A,B) \subset F$ が成り立つ.
 
 $F_1 = E(A,B)$ はボンドであるから,(i)で示したことにより,任意の全域木と交わる.$F_1\subset F$ と $F$ の極小性より $F = F_1 = E(A,B)$ である.
 よって問題~1.36 より,$F$ はボンドである. 
\end{enumerate}