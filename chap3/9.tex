\subsection{}
(i), (ii) を合わせて $|G|$ に関する帰納法により示す.$|G|\geq 4$ とし,より小さい場合に示されたとする.

まず $G$ がサイクルとする.$G\neq K^3$ より長さ $4$ 頂点のサイクルである.このとき,任意の $e$ に対して $G/e$ が $2$ 連結となるのでよい.
 
$G$ がサイクルでないとする.耳分解より,$2$ 連結部分グラフ $H$ および $a,b\in H, a\neq b$ を端点とする $H$ パス $P$ であって $G = H + P$ を満たすものが存在する.
$P$ の長さを $n$ とする.

\paragraph{(i)の証明}
$e\in P$ であるとする.
$n = 1$ ならば,$G-e=H$ となるのでよい.$n = 2$ かつ $a,b\in H$ が隣接しているならば,$G/e \cong H$ となるのでよい.
$n = 2$ かつ $a,b\in H$ が隣接していないならば,$G/e$ は,$H$ に $H$ パスをひとつ加えたものなので,耳分解より $2$ 連結である.
$n\geq 3$ の場合も同様である.

$e\notin P$ であるとする.$H = K^3$ のときは,$e = ab$ ならば $G - e$ が,そうでなければ $G / e$ がサイクルとなるのでよい.そうでないとき,帰納法の仮定より
$H-e$ または $H/e$ は $2$ 連結である.$H\neq K^3$ であれば帰納法の仮定より $H-e$ または $H\setminus e$ が $2$ 連結である.
$e\neq ab$ ならば,$G-e$, $G/e$ はそれぞれ $H/e$, $G\setminus e$ に耳を付け加えたものと同型であるから,どちらかは $2$ 連結である.
$e = ab$ ならば,$G - e$ は $H$ の辺を細分したものと同型であるから,$G-e$ に対する耳分解の存在と $H$ に対する耳分解の存在は同値であり,$G-e$ も $2$ 連結であることが分かる.

以上により (i) が示された.

\paragraph{(ii)の証明}
$H = K^3$ であるとする.$ab$ と異なる辺 $e\in H$ をとれば,$G/e$ はサイクルなので $2$ 連結である.よって $H=K^3$ のときはよい.

$H\neq K^3$ であるとする.帰納法の仮定より,$e\in H$ であって $H/e$ が $2$ 連結であるものが存在する.
$e\neq ab$ であれば,$G/e$ は $H/e$ に耳をひとつ付け加えたものと同型なので,耳分解より $2$ 連結である.

$e = ab$ であるとする.このとき,$n\leq 2$ である.$P$ 上から任意に辺 $e$ をとれば,
$G/e$ は $H$ または $H$ に耳をひとつ付け加えたものと同型なので,$2$ 連結である.