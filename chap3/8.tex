\subsection{}
$e \in E(G)$ とし,縮約を $\pi\colon G\longrightarrow G/e$ と書く.
$X\subset G/e$ に対して $\overline{X} = \pi^{-1}(X)$ とする.
このとき,$a,b \in G\setminus X$ に対して次は同値である.
\begin{itemize}
 \item $G\setminus \overline{X}$ において $ab$ パスが存在する.
 \item $(G/e)\setminus X$ において $\pi(a)\pi(b)$ パスが存在する.
\end{itemize}

このことを踏まえて,問題の主張を示す.
$G/xy$ が $k$ 連結であることは,次がともに成り立つことと同値.
\begin{itemize}
 \item $|G/xy| \geq k + 1$.
 \item 任意の $k-1$ 元部分集合 $X\subset G/xy$ に対して,$(G/xy)\setminus X$ は連結.
\end{itemize}
これらはさらに次と同値.
\begin{itemize}
 \item $|G| \geq k + 2$.
 \item 任意の $k-1$ 元部分集合 $X\subset G/xy$ に対して,$G\setminus \pi^{-1}(X)$ は連結.
\end{itemize}
$G$ は $k$ 連結であるから,$|\pi^{-1}(X)| \leq k-1$ の場合には後者の主張は成り立つ.
したがって,$|\pi^{-1}(X)| = k$ の場合の主張と同値.これは,$x,y\in X$ の場合に対応し,$\pi^{-1}(X)$ は
$x,y$ を含む $k$ 元集合を動く.よって次のように同値変形される.
\begin{itemize}
 \item $|G| \geq k + 2$.
 \item $x,y$ を含む任意の $k$ 元部分集合 $X\subset G$ に対して,$G\setminus X$ は連結.
\end{itemize}
$G' = G\setminus\{x,y\}$ とすると,これは次と同値であることは容易に分かる.
\begin{itemize}
 \item $|G'| \geq k$.
 \item 任意の $k-2$ 元部分集合 $X\subset G'$ に対して,$G'\setminus X$ は連結.
\end{itemize}
これは $G'$ が $k-1$ 連結であるという主張に他ならない.