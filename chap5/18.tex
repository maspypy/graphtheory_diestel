\subsection{}
$G$ が木の場合に $P_G(k) = k(k-1)^{n-1}$ が成り立つことは,根から色を決めていくことで容易に分かる.
逆を示す.

まず,次を示す.
\begin{enumerate}
 \item $|G| \geq 1$ のとき,$P_G(k)$ は $k$ で割り切れる.
 \item $G = G_1\amalg G_2$ のとき,$P_G(k) = P_{G_1}(k)P_{G_2}(k)$ が成り立つ.
\end{enumerate}
(ii) は容易である.(i) は,頂点$v\in G$ をひとつとると,$v$ の色が $1,2,\ldots,k$ であるような彩色が同数あることから分かる.

$G$ が $P_G(k) = k(k-1)^{n-1}$ を満たすとする.上で示したことより,$G$ は連結である.$G$ の全域木 $T$ がとれる.
$P_G(k) = P_T(k)$ であるから,$G$ は $T$ と同数の彩色を持つ.これは $T$ の彩色が常に $G$ の彩色でもあるということと同値である.
$a,b$ を $T$ において隣接していない $2$ 頂点とする.このとき,$T$ の彩色であって $a,b$ に同じ色が与えられているものが存在する
(この $2$ 頂点のみを色 $1$ で塗り,他の頂点をすべて異なる色で塗ればよい).
この彩色が $G$ の彩色でもあることから,$ab\notin G$ である.したがって $G=T$ が示され,$G$ は木グラフである.