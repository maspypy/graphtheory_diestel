\subsection{}
次を確かめておく.
\begin{enumerate}
 \item[(a)] 準同型の合成は準同型.
 \item[(b)] $k$ 色以下の彩色の存在と,$K^{k}$ への準同型の存在は同値.
 \item[(c)] $n\geq 3$ に対して準同型 $C_{n+2}\longrightarrow C_n$ が存在.
\end{enumerate}

(a) は明らか.

(b) を示す.彩色の存在は,写像 $c\colon V\longrightarrow \{1,2,\ldots,k\}$ であって
$ab\in E(G)\implies c(a)\neq c(b)$ となるものの存在と同値.$\{1,2,\ldots,k\}$ を $K^{k}$ の頂点と同一視する.
$c(a)\neq c(b)$ というのは,$K^k$ において $c(a)$ と $c(b)$ が隣接していることと同値.
よって彩色の存在は準同型の存在と同値である.

(c) を示す.$C_n$ の頂点集合をそれっぽく $0,1,\ldots,n-1$ と同一視する.
$f\colon C_{n+2}\longrightarrow C_n$ を,
\[
 f(v) = \begin{cases} v & (v < n)\\ 0& (v = n) \\ 1 & (v = n + 1)\end{cases}
\]
と定めればよい.

問題の解答にうつる.

(i) は (a) そのものである.

(ii) を示す.二部グラフは $2$ 色以下での彩色の存在と同値であるから,$K^2$ への準同型の存在と同値である.
$C_{2n}$ は二部グラフなので,$C_{2n}$ への準同型があれば $K^2$ への準同型もあるのでよい.

(iii).
\begin{itemize}
 \item $C_{17}$ から $C_7$ への準同型は存在する.(c) と (a) から分かる.
 \item $C_7$ から $C_{17}$ への準同型は存在しない.存在したとすると,$C_{17}$ における長さ $7$ の閉歩道がとれる.
 $0\in C_{17}$ から長さ $7$ の歩道で到達できる頂点は $7,5,3,1,16,14,12,10\in C$ のどれかである.閉歩道にはならない.
 \item $C_{16}$ から $C_7$ への準同型は存在する.$C_{16}\longrightarrow C_{14}$ と $C_{14}\longrightarrow C_7$ を合成することでつくれる.
 $C_{14}\longrightarrow C_7$ は $n\longmapsto n\bmod 7$ により構成できる.
 \item $C_{17}$ から $C_6$ への準同型は存在しない.$C_{17}$ は奇サイクルを含み,二部グラフではないので.
\end{itemize}