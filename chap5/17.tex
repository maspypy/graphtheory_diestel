\subsection{}
\paragraph{解法1}
$e = ab$ がグラフ $G$ の辺であるとする.
$G - e$ の彩色 $c\colon V(G)\longrightarrow [k]$ を,
$c(a) = c(b)$ を満たすか否かで場合分けして数えることで,
$P_{G-e}(k) = P_{G}(k) + P_{G/e}(k)$ が成り立つことが分かる.

したがって $P_G(k) = P_{G-e}(k) - P_{G/e}(k)$ が成り立つ.

あとはこのことを用いて,$\|G\|$ に関する帰納法により証明できる.
base case は辺がない場合で,$n = |G|$ とすれば $P_G(k) = k^n$ である.

\paragraph{解法2}
$Q_G(k)$ を,彩色 $c\colon V(G)\longrightarrow [k]$ かつ全射であるものとする.
$P_G(k)$ を色数により場合分けして計算することで,
\[
 P_G(k) = \sum_{i=0}^{\infty}\binom{k}{i}Q_G(k)
\]
が分かる.さらに明らかに $n > |G| \implies Q_G(k) = 0$ であるから
$P_G(k) = \sum_{i=0}^{n}\binom{k}{i}Q_G(k)$ となる.
この表示より,$P_G(k)$ が $n$ 次以下の多項式であることが分かる.$n, n-1$ 次の係数を計算するためには
$Q_G(k), Q_G(k-1)$ が求められればよく,これは容易である.
