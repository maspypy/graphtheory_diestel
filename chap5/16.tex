\subsection{}
\paragraph{(i) $\implies$ (iii)}
$k$ 色以下の彩色 $c$ をとり,辺 $e = uv$ の向きを,$c(u) < c(v)$ であるか否かによって $u$ から $v$,$v$ から $u$ と定めれば,
条件を満たすことは容易に確かめられる.

\paragraph{(iii) $\implies$ (ii)}
自明.

\paragraph{(ii) $\implies$ (i)}
(ii) の向き付けを固定し,$G$ は有向グラフであるとして扱う.
$H\subset G$ を,$G$ の部分グラフかつDAGであるもののうち極大なものとする.

$H$ の topological 順序 $(v_1,v_2,\ldots)$ をひとつ固定する.次の要領で貪欲に着色 $c$ を構成する:
\[
 c(v) = \max\{c(u)\mid uv \in H\} + 1.
\]
ただし,$u$ が存在しない場合には $c(v) = 0$ とする.

次の性質が容易に確かめられる:
\begin{enumerate}
 \item[(a)] $uv\in H\implies c(u) < c(v)$.
 \item[(b)] $c(v) > 1 \implies \exists u, uv\in H, c(v) = c(u) + 1$.
\end{enumerate}

性質 (b) より,各頂点 $v$ に対して $v$ を終点とする長さ $c(v)$ の有向パスが存在する.
(ii) の仮定より $c(v) < k$ である.よって着色 $c$ は $k$ 色以下からなる.これが彩色であることを確かめればよい.

性質 (a) より,$H$ に含まれる有向辺 $uv$ については端点の色は異なる.
$uv\in E(G)\setminus E(H)$ とする.$H$ の極大性より,有向 $vu$ パスが存在する.したがって性質 (a) より $c(v) < c(u)$ であるから
この場合も $u,v$ の色は異なる.
以上により示された.