\subsection{}
$\chi(G) = k$ かつ三角形を含まないグラフ $G$ をもとにして,$k+1$ の場合のそれを作る.
$G = \{v_1,\ldots,v_n\}$ とする.

$\overline{G}$ を以下のようにつくる:
\begin{itemize}
 \item $V(\overline{G}) = \{v_1,\ldots,v_n\} \amalg \{u_1,\ldots,u_n\} \amalg \{w\} = V \amalg U\amalg \{w\}$.
 \item $V$ の頂点間は,$G$ と同様に辺を張る.
 \item $v_iv_j\in E(G)$ であるか否かにしたがって,$v_iu_j$ に辺を張る・張らない.
 \item $U$ の頂点間には辺を張らない.
 \item $V$ と $\{w\}$ の間には辺を張らない.
 \item 任意の $u\in U$ に対して $u$ と $w$ の間に辺を張る.
\end{itemize}

{\Huge 図!}

次を示せばよい.

\begin{enumerate}
 \item[(a)] $\overline{G}$ は三角形を含まない.
 \item[(b)] $\chi(\overline{G})\leq k+1$.
 \item[(c)] $\chi(\overline{G})\geq k+1$.
\end{enumerate}

\paragraph{(a)の証明}
$w$ は $U$ の点としか隣接せず,$U$ の点どうしは隣接しないので,$w$ を含む三角形は存在しない.
$G$ に対する仮定より,$V$ の点だけからなる三角形は存在しない.
また,$U$ の点同士は隣接しない.よって,$v_iv_ju_k$ の形の三角形が存在しないことを示せばよい($v_i,v_j\in V, u_k\in U$).
$v_iv_ju_k$ が $\overline{G}$ の三角形であるとすると,$v_iv_jv_k$ は $G$ の三角形となるので示された.

\paragraph{(b)の証明}
$c\colon G\longrightarrow \{1,2,\ldots,k\}$ を $G$ の $k$ 彩色とする.
$\overline{c}\colon \overline{G}\longrightarrow \{1,2,\ldots,k+1\}$ を次のようにつくる:
$\overline{c}(v_i) \overline{c}(u_i) = c(v_i)$, $\overline{c}(w) = k+1$.これが彩色になることは容易に確かめられる.

\paragraph{(c)の証明}
$\overline{c}\colon \overline{G}\longrightarrow \{1,2,\ldots,k\}$ を $k$ 彩色が存在したと仮定.
$\overline{c}(w) = k$ としてよい.
$c(v_i) = \begin{cases}\overline{c}(v_i) & (\overline{c}(v_i)\neq k)\\ \overline{c}(u_i) & (\overline{c}(v_i) = k)\end{cases}$ により
$c\colon G\longrightarrow \{1,2,\ldots,k-1\}$ を定めると,$G$ の彩色になる.$\chi(G) = k$ に矛盾.


\noindent\textbf{参考:}Mycielski, J. (1955), “Sur le coloriage des graphes”, Colloq. Math. 3: 161–162. 