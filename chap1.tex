\subsection{}
$2|E| = \sum_v \deg(v)$ より $|E| = \frac12 n(n-1)$. 

\subsection{}
以下,内周と外周については$d\geq 2$ とする.

このグラフを $G$ として,$d(G) = d$, $\|G\| = \frac{1}{2}\cdot d\cdot 2^d$, $\diam(G) = d$, $g(G) = 4$, 外周は $2^d$.

外周の計算はHamiltonサイクルの存在証明と同値.
「$C^n \times \{0,1\}$」 が $C^{2n}$ を含むことから($d=2$ をbase stepとして)帰納法が使える.

\subsection{}
$C$ の長さが $\sqrt{k}$ 以上であれば示すことはない.
$C$ の長さが $\sqrt{k}$ 未満であれば、問題のパス $P$ が $C$ と交わる回数も $\sqrt{k}$ 以下である.
交点から交点までパスに分解すると,どれかのパスは $\sqrt{k}$ 以上の長さを持って,$C$ と合わせて求めたいサイクルが得られる.


\subsection{}
Yes.$G = C^{2k+1}$ のときに $g(G) = 2k+1$, $\diam(G) = k$ となって等号.

\subsection{}
BFS木の性質.容易である.

\subsection{}
任意のグラフとあるが,$G\neq \emptyset$ が必要である.
中心をとって($G\neq \emptyset$ よりとれる)議論すれば容易.

\subsection{}
$g = 2k+1$ あるいは $g = 2k + 2$ とおく.
$g$ が奇数ならば $1$ 点,偶数ならば隣接する $2$ 点を固定.
これらの点から距離 $n$ の点集合を $D_n$ とする.

$1\leq n < k$ ならば $v\in D_n$ の近傍 $N(v)$ は $D_{n-1}\cup D_n \cup D_{n+1}$ に含まれる.
$D_{n-1}$ に少なくとも $1$ 点の近傍がある.$D_{n-1}\cup D_n$ に $2$ 点あると $g$ より小さな閉歩道ができて矛盾.
$N(v)$ の元は,唯一の $D_{n-1}$ の点および $D_{n+1}$ の点からなることが分かる.

特に,$d-1$ 個以上の $D_{n+1}$ の点と接続する.また再び内周の議論から $u,v \in D_n$ に対して
$N(u)\cap N(v) = \emptyset$ がいえるのでこれらの点は $v$ ごとにdisjoint.
$|D_{n+1}|\geq (d-1) |D_n|$がいえる.

あとは $D_0$ が $1$, $2$ 点であることを使って下から評価すればできる.

\subsection{}

\subsection{}

\subsection{}

\subsection{}

\subsection{}

\subsection{}

\subsection{}

\subsection{}

\subsection{}

\subsection{}

\subsection{}

\subsection{}

\subsection{}

\subsection{}

\subsection{}

\subsection{}

\subsection{}

\subsection{}

\subsection{}

\subsection{}

\subsection{}

\subsection{}

\subsection{}

\subsection{}

\subsection{}

\subsection{}

\subsection{}

\subsection{}

\subsection{}

\subsection{}

\subsection{}

\subsection{}

\subsection{}

\subsection{}

\subsection{}

\subsection{}

\subsection{}

\subsection{}

\subsection{}

\subsection{}

\subsection{}

\subsection{}

\subsection{}

\subsection{}

\subsection{}

\subsection{}

\subsection{}

\subsection{}

\subsection{}

\subsection{}

\subsection{}

\subsection{}

