\subsection{}
まず次の主張を示す:

$G$ を平面的グラフで,$3$ 点 $a,b,c\in G$ は三角形をなすとする.
このとき,$G$ の線分からなる描画であって,外面が三角形 $abc$ であるものが存在する.

$|G|$ についての帰納法により示す.$|G| = 3$ のときはよい.
$|G|\geq 4$ とする.必要であれば辺を追加することで,$G$ は平面性のもと辺極大であるとしてよい.
したがって $G$ は三角形分割であるとしてよい(命題4.2.8).

三角形分割において $\|G\| = 3|G| - 6$ である.したがって $\sum_{v\in G}(6 - \deg_G(v)) = 12$ である.
このことと $|G|\geq 4$ から,$\deg_G(v)\leq 5$ となる $v\notin \{a,b,c\}$ の存在が分かる.$n = \deg_G(v)$ とする.

帰納法の仮定を用いると,$H = G - v$ の線分からなる描画であって,外面が $abc$ であるようなものが存在する.
$v$ の近傍を $v_1,\ldots,v_n$ とすると,$H$ において $P = v_1\ldots v_n$ はひとつの面となる.
また,$abc$ が外面であることから $v$ は $P$ の内部に存在し,$G$ は,$H$ の $n$ 角形 $P = v_1\ldots v_n$ を $n$ 個の三角形 $v_kv_{k+1}v$ ($1\leq k\leq n$) で置き換えることにより得られる.

$n\leq 5$ より,$n$ 角形 $P$ の内部の点 $v$ であって,線分 $vv_i$ が $P$ の内部に含まれるものが存在する\footnote{参考:\url{https://en.wikipedia.org/wiki/Art_gallery_problem}}.
そのように $v$ をとって,$H$ の描画に対して頂点 $v$ および線分 $vv_k$ をつけ加えることで,$G$ の線分描画が得られる.

以上により冒頭の主張が示された.
任意の $G$ に対する主張を示す.$|G|\geq 3$ としてよく,辺極大としてよいので,既に示したことからよい.


