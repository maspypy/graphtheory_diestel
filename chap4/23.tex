\subsection{}
まず,$G$ が外平面的であることと,$G * \pt$ が平面的であることは同値であることを示す.

$G$ が外平面的であるとき,$G$ の外平面的な描画をとり,$\pt$ を外側領域にとることで,$G*\pt$ の平面的な描画が得られる.
逆に $G * pt$ が平面的であるとする.$H$ を $G * \pt$ に辺を加えて三角形分割にしたものとする.$\pt$ の近傍全体はサイクルをなし,
$v$ はその内部または外部にある.必要なら同相写像でうつすことで,外部にあるとしてよい.この描画を $G$ に制限することで,
$G$ の外平面的な描画が得られる.

以上の考察のもと,問題の解答にうつる.

\paragraph{外平面的ならば $K^4$ および $K_{2,3}$ をマイナーに持たないこと}
対偶を示す.$G$ が $K_4$ または $K_{2,3}$ をマイナーに持つとする.このとき,
$G * \pt$ は $K_4 * \pt$ または $K_{2,3} * \pt$ をマイナーに持つ.
$K_4 * \pt$ は $K^5$ と同型,$K_{2,3} * \pt$ は $K_{3,3}$ を部分グラフに持つので,いずれの場合も
$G * \pt$ は $K^5$ または $K_{3,3}$ をマイナーに持つ.したがって Kuratowski の定理より
$G * \pt$ は平面的ではない.したがって $G$ は外平面的ではない.

\paragraph{$K^4$ および $K_{2,3}$ をマイナーに持たないならば,外平面的であること}
対偶を示す.$G$ が外平面的でないとする.これは $G * \pt$ が平面的でないことと同値である.
Kuratowskiの定理より,$G * \pt$ は $K^5$ または $K_{3,3}$ をマイナーに持つ.
$\pt$ の寄与を除くことで,$G$ は $K^4$ または $K_{2,3}$ をマイナーに持つことが分かる.